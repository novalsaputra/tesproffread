\documentclass{iaesarticle}

%%required package. add for your convenient, but do not remove the initial
\setlength{\headsep}{0.15in}
\usepackage{amsmath, amsfonts, amssymb, float, fancyhdr}
\usepackage[figuresright]{rotating}
\usepackage{authblk, graphicx, indentfirst, lastpage, lipsum}
\usepackage{pifont}
\renewcommand{\Authsep}{, }
\renewcommand{\Authand}{, }
\renewcommand{\Authands}{, }
\setlength{\affilsep}{0cm}
\renewcommand\Authfont{\normalsize}
\renewcommand\Affilfont{\normalfont\small}
\makeatletter
\renewcommand{\@biblabel}[1]{[#1]\hfill}
\makeatother
\usepackage{subfig, caption, epstopdf}
\usepackage[left=3cm, right=2.5cm, top=2.5cm, bottom=2.5cm, includehead, includefoot]{geometry}
\usepackage[justification=centering]{caption}
\captionsetup{labelsep=period}
\usepackage{titlesec}
\usepackage{xcolor}
\titleformat{\section}
{\normalfont\normalsize\bfseries\uppercase}{\thesection}{1.7em}{}
\titleformat{\subsection}
{\normalfont\normalsize\bfseries}{\thesubsection}{.95em}{}
\titleformat{\subsubsection}
{\bfseries}{\thesubsubsection}{.2em}{}
\titlespacing*{\section}{0cm}{0.7cm}{0cm}
\usepackage{enumitem}
\usepackage[numbers,compress]{natbib}

\usepackage[backref]{hyperref}
\usepackage{multirow}
\usepackage{float}




%%leave copyright info to the editor
\CopyrightLine[]{}{\textit{This is an open access article under the \textcolor{blue}{\underline{CC BY-SA}} license.}
\vspace{.5em}}

%%author


%%author's affiliation



%%title and shortitle (for footer)
\title{Triclustering method for finding biomarkers in human immunodeficiency Virus-1 gene expression data}
\shorttitle{Triclustering method for finding biomarkers in human immunodeficiency Virus-1 gene expression data (Titin Siswantining)}

%%starting
\begin{document}
\setcounter{page}{1}

%%indentation. do not change
\setlength{\parindent}{1.27cm}

%%header and footer setting. do not change
\pagestyle{fancy}
\fancyhfoffset{0cm}

%%journal info
\journalname{}
\journalshortname{}
\journalhomepage{}
\vol{}
\no{}
\months{}
\years{}
\issn{}
\DOI{}
\pagefirst{}
\pagelast{}

%%build title
\maketitle

%%border setting. do not change
\hrule
\vspace{.1em}
\hrule
\vspace{.5em}
\noindent
\parbox[t][][s]{0.275\textwidth}{%
\textbf{Article Info}
\vspace{.5em}
\hrule
\vspace{.5em}

\vspace{.5em}

%%article info. editor's privilege
%%Received Jan 20, 2018

%%Revised Nov 9, 2018

%%Accepted Jun 10, 2018

\vspace{.7em}

\vspace{.5em}
\hrule
\vspace{.5em}

\vspace{.5em}
%%write keyword here. separate by \sep
\vspace{.5em}

\vspace{\fill}
}
\parbox{0.025\textwidth}{\hspace{0.5em}}
\parbox[t][][s]{0.7\textwidth}{%
\begin{abstract}
\vspace{.3em}
%% Text of abstract
HIV-1 is a virus that destroys CD4 + cells in the body's immune system, causing a drastic decline in the immune system. Analysis of HIV-1 gene expression data is urgently needed. The technology used to analyze gene expression data is microarray. Microarray technology is used to measure the expression value of thousands of genes in various conditions. In the gene expression series data which is formed in three dimensions, the analysis used is triclustering. Triclustering is an analysis technique on 3D data that aims to group data simultaneously on rows and columns across different times/conditions. The result of this technique is called a tricluster. Tricluster is a subspace in the form of a subset of rows, columns, and time/conditions. In this study, we used the $\delta$-Trimax, THD Tricluster and MOEA methods by applying different measures, namely transposed virtual error, New Residue Score, and Multi Slope Measure. Therefore, this study using gene expression data consisting of 22,283 probe gene id, 40 observations, and four conditions, namely normal, acute, chronic, and non- progressor. Tricluster evaluation was carried out using Inter Temporal Homogeneity. Through this Triclustering process, an analysis of the probe id gene that affects AIDS is carried out. From this analysis, a gene symbol associated with AIDS due to HIV-1 was found in every condition of normal, acute, chronic, and non-progressive HIV-1 patients, is HLA-C.
\end{abstract}
}
\parbox[l]{\textwidth}{%
\rule{0.275\textwidth}{0.5pt} \hspace{0.5cm} \hrulefill
\\
\emph{\textbf{Corresponding Author:}}
\vspace{.5em}\\
%% correspondence info. separate by \\
Titin Siswantining,\\
Department of Mathematics,\\
Universitas Indonesia,\\
Depok,Jakarta, Indonesia \\
Email: titin@sci.ui.ac.id
}
\vspace{.5em}
\hrule
\vspace{.1em}
\hrule


%% main text

\section{Introduction}

At present, there are many studies on gene expression, there are several technologies that can help in the process of measuring gene expression. The technology commonly used is Microarray Technology. Microarray technology is used to measure thousands of gene expression simultaneously, becomes a very large data source, is formed in a structured manner and is called the gene expression matrix. In recent years, microarray technology has been used to measure the expression value of thousands of genes under a wide variety of experimental conditions at different points in time. Such a dataset can be referred to as gene-sample-time (GST) expression data. In addition to the gene-sample-time expression data, there are other three-dimensional data used in the biomedical and social fields. Other examples of three-dimensional data are individual-feature-time, node-node-time data or commonly referred to as attribute-data-context.

One of the techniques often used for gene expression data analysis is clustering, the analyzed gene expression data is in the form of a matrix with rows representing the probe gene id and columns representing observations. Clustering is a grouping technique that works on two-dimensional (2D) data, which aims to group rows or columns on a matrix to obtain a sub-matrix from one of them. Furthermore, this technique was developed into biclustering with the aim of grouping 2D data in the form of rows and columns in a matrix simultaneously to obtain a submatrix from both of them. As data grew larger and more complex, and there were constraints in classifying gene expressions that had a certain time/ condition, this technique was further developed into tricluster. Triclustering aims to group three-dimensional (3D) data in rows and columns across different times/conditions, where the result of this technique is called a tricluster. Tricluster is a subspace in the form of a subset of rows, columns, and time/conditions .

In 2018, Kakati researched triclustering using a method called THD-Tricluster. This method consists of two steps, namely generate biclusters and generate triclusters. This method analyzes 3D gene expression data and can identify various patterns such as absolute, shifting, scaling, as well as shifting and scaling using the Shifting and Scaling Similarity (SSSim) measure. The SSSim measure was used to overcome the limitations of several previous studies which, identified only one pattern and not shifting and scaling patterns. This research is a new thing because it can mine data in various patterns, especially shifting and scaling. Also, the algorithm used by Kakati to build a tricluster created in MATLAB software. Apart from the (shifting and scaling similarity) SSSim measure, there are also new residue scores and transposed virtual error, which is a measure that handles various patterns, especially shifting and scaling. This measure works by looking for correlations between rows and columns of a matrix known as Pearson correlation. In this study, researchers were inspired to implement the THD-Tricluster method by using a new residue score on the 3D gene expression data for HIV-1 disease. HIV-1 is a deadly disease that requires analysis to inhibit disease progression because late diagnosis causes many HIV-1 sufferers to die. In 2006, Sushmita Mitra and Haider Banka introducing bicluster search phase uses the Multi-Objective Evolutionary Algorithm with Mean Square Residue (MSR) measurement. Then build a tricluster with the Trigen algorithm.

In this study we use the THD tricluster method with a new residue score, Transposed Virtual Error, and use multi-objective (MOEA) as an optimization in the search for bicluster, as well as the Trigen algorithm through Multi-Slope measurement. Then the results are also compared with the $\delta$-Trimax method which uses the mean square residual measure. The tricluster obtained were evaluated with inter-temporal homogeneity and analyzed the tricluster results to find biomarkers against Human Immunodeficiency Virus-1 (HIV-1). Open source R-based software is used by researchers to implement the program used to build a tricluster with a Multi-Objective approach to Human Immunodeficiency Virus-1 gene expression data.

 
%%%%%%%%%%%%%%%%%%%%%%%%%%%%%%%%%%%%%%%%%%
\section{Materials and Methods}
\subsection{Data Sets}

This research data uses developmental data for HIV-1 referred to in and obtained from the National Center for Biotechnology Information (NCBI) website with the website address with ID GSE6740. NCBI is a site that provides access to biomedical and genome information. The data used by consisted of 22283 gene ID probes, 10 observations and 4 conditions. These conditions include individuals with HIV Acute sufferers, HIV Chronic sufferers, HIV nonprogressors and uninfected individuals. The four data have different gene expression values. Before using the triclustering method, the gene ID probe selection process was carried out and 2557 gene ID probes were produced and 10 observations for each condition.

\subsection{THD-Tricluster}

THD-Tricluster is a triclustering method capable of mining 3-dimensional data with absolute patterns, shifts, scaling, shifting and scaling, with high biological significance. The THD-Tricluster method consists of two stages, namely, generate biclusters and generate triclusters. The THD-Tricluster workflow with three-time points can be seen in

\subsubsection{Generate Biclusters with Modifed CC Algorithm}
The CC algorithm is the first algorithm used by to find biclustering in gene expression data using the Mean Square Residue (MSR) as a measure. In this research, the CC algorithm is modified by using the Transposed Virtual Error $VE^t$   measure, which can group data based on shifting patterns and scaling. The CC algorithm used consists of three phases, namely Multiple Node Deletion, Single Node Deletion, and Single Node Addition phases.

\paragraph{Multiple Node Deletion Phase}
The input in the Multiple Node Deletion phases is a matrix of gene expression data. In this CC's algorithm, we need the alpha ($\gamma$) parameter. These parameters are the limits of the selected data. This phase is the phase to delete rows and columns simultaneously if they do not meet the chosen parameters.

\paragraph{Single Node Deletion Phase}
The input of this phase is the submatrix obtained from the multiple node deletion phases. The parameter used is the only delta ($\delta$). If in the multiple node deletion stage rows and columns can be deleted simultaneously, then in this phase, the columns and rows are only deleted one by one. The deletion of rows and columns occurs by comparing the Transposed virtual error value with delta ($\delta$). Rows and columns that select deleted, several steps are needed, namely calculating the average row value of the matrix element minus the virtual condition ($m_r$), and the average column value of the matrix element minus the virtual condition ($m_c$), then finding the max value max($m_{r_i} ,m_{c_j }$). Columns or rows that have a greater value are deleted. This process takes place until it meets the selected delta parameter value.

\paragraph{Single Node addition Phase}  
In this phase, the parameter used is the delta ($\delta$), and the input of this phase is a matrix obtained from the single node addition phase. The matrix is added again with rows or columns that have been deleted in the multiple node deletion phase. This is done to produce a more optimal bicluster by adding a column or row so that the transposed virtual error value obtained is close to the limit of the selected parameter value.

\paragraph{Lift Algorithm}
The lift algorithm consists of two phases, Single Node Deletion, and Single Node Addition. Single node deletion algorithm aims to remove nodes in the form of rows or columns on the matrix that have values exceeding the threshold. In this study, the value in question is the result of the calculation of the new residue score ($S$). Iteration of this step is complete until $S \leq \delta$ . The submatrix that has been obtained from Single Node Deletion may not be maximal, so we check again on the deleted nodes in the single node deletion step with single node addition step. Checking is done by recalculating the value of $S$ in rows or columns that have been deleted on condition that it can maintain an $S$ value that is less than node so that nodes that produce a small $S$ value will be added. This iteration will end when no more nodes can be added.
\subsubsection{Generate Tricluster}
The generate triclusters stage aims to produce a set of tricluster. This stage finds the tricluster by finding the slices of all bicluster combinations for each condition of the generate biclusters stage.

\paragraph{New Residue Score}
The principle of the new residue score is to find the correlation between rows and columns of a matrix known as the Pearson correlation. This correlation indicates the degree of the linear relation between two vectors and will produce a perfect bicluster correlation.

The result of this correlation is a number between 1 and -1, which means that there is a perfect positive (or negative) linear relationship between the indexed submatrices. The correlation value in the indexed submatrix column ($I$, $J$) is defined as:

The values of $S_{col}$ and $S_{row}$ indicate the degree of correlation of the respective columns and rows of a submatrix. The correlation values of the indexed submatrix ($I, J$) are defined as follows:


Given a submatrix indexed ($I,J$) the correlation value (residue) of the submatrix is:

The lower the correlation value, the better the column and row correlation will be. According to, the determination of the bicluster residual value lies at the minimum value of row and column correlation.

\paragraph{Transpose Virtual Error}
The basis behind $VE$ is to measure the genes that have a general tendency in bicluster. To find out the general trend of this gene across the conditions contained in bicluster, new virtual lines are counted from the Bicluster gene, which is called a virtual pattern or virtual $\rho$ gene. Each element $\rho_j$ of $\rho$ is calculated as the average of the $jth$ column or experimental conditions, as in the equation:

In the $VE^t$ the calculation, a standardization process is carried out on each data element and the virtual conditions. The process of standardizing the elements of each submatrix is by calculating the average value for each row and calculating the standard deviation value for each row which can be expressed as follows:

Meanwhile, the virtual standardization process can be calculated by calculating the average of each column averages and calculating the standard:

and the $VE^t$ value for the submatrix $\mathfrak{B}$ is obtained using the standardization of the virtual condition value $\hat{\rho}$ together with the standardized sub-matrix elements $\hat{b}_{ij}$. Here is the formula for calculating the $VE^t$ value:

$VE^t$  has been shown to be zero for Biclusters with perfect shifting, scaling, or compound patterns. Therefore, it is efficient to recognize shifting and scaling patterns in Biclusters either simultaneously or independently.

\paragraph{Multi Slope Measure (MSL)}
To understand MSL needs to be established graphic representation tricluster $TRI_{xop}$,with $x$, $o$ and $p$ is one of gene $G$, condition $C$ and time or depth $T$ so that element $x$ on $TRI_{xop}$ will be on $X$-axis and element $o$ on $TRI_{xop}$ will represent outline as much as the panels which are elements $p$ on $TRI_{xop}$ as shown in . MSL calculate the difference among the angles formed by every series traced on each of three graphic representation taking into account $TRI_{gct}$, $TRI_{gtc}$, and $TRI_{tgc}$. MSL taking into account the effect of adjancent points in time. Can be observed display examples $TRI_{gtc}$ from $TRI$ = $G{g_1, g_4, g_7, g_10}, C{c_2, c_5, c_8}, T{t_0, t_2, t_{11}}$ on . Can be seen how each outline or gene forms a set of angles (two for this case) determined by each point in time on the X-axis for each panel or experimental condition.

MSL measures the average difference between the angles formed by the probe-id gene in all row and columns for each individual or candidate tricluster. To calculate the MSL measure of a tricluster, first perform a multiangular ratio calculation. Defined the $AC_{multi}$ tricluster $TRI_{tgc}$ as the average of difference $\Delta$ from an angle vector $av_{op} \in ang set$ from all of the outline $o$, for each panels $p$ $(V_{mc})$ and the same for the rest of panels $(H_{mc})$, with $N_{mc}$ is the number of differences that are formed. All the angular vectors of an outline $o$ on panel $p$ is defined as a set of angles formed by outline $o$ taking into account each data point on the $X$-axis , each outline will have (many axis mark $X$–$1$ ) angle as can be seen in the . Difference $\Delta$ among two vector angles $av_A$ and $av_B$ is defined as the average of $MAX - MIN$ ($MAX$ is the maximum and $MIN$ is the minimum from two angles $av_{A}(i)$ and $av_{B}(i)$) of any component or angle $i$ from $av_A$ and $av_B$.

$AC_{multi}$ based on multiple operations with $av_{op}$ angle vector. These elements are obtained based on the concept of a series so the series $S_{op}$ from the ouline $o$ for each panel $p$ is the set of value pairs from the $X$-axis ($x_i$) and expression level ($el_j$) which forms the outline. For each set $S_{op}$,angle of alpha $a_{x_i}$ count as spin arctangent of the slope of the line formed by the points ($x_i,el_i$) and ($x_{next(i)},el_{next(i)}$). The spin operations from an angle is the positive equivalent of angle if it is negative.


To conclude, the MSL measure of a TRI tricluster as shown in equation 23 is the mean of the angular comparisons of three graphical representations of a tricluster.

\subsection{Multi-Objective Optimization}
Multi-Objective Optimization is a development of a genetic algorithm. The genetic algorithm is used to solve problems in the search for optimization values. Multi-Objective Evolutionary is based on genetic processes in living things, with the process of developing generations in a population which is gradually based on natural selection in the ability to survive. This algorithm is used to search for optimal bicluster and tricluster.

\subsubsection{Initial Population}
This initialization of the population is the first step in multi-objective optimization. Population is the number of individuals who represent the desired solution. In this study, individuals are the number of submatrices in the first method (MOEA and THD tricluster) and subspaces in the second method (Triclustering Genetic Based) obtained from the gene expression matrix. The initial population is generated randomly so that an initial solution is obtained. Initialization is done by coding. Coding is an important aspect of multi-objective optimization. Encoding is a technique for expressing the initial population as a potential solution to a problem onto an individual chromosome. In this study, researchers used binary coding. Binary coding is used to encode (encode) all chromosomes. Each possible bicluster or tricluster is represented by the total bits of the binary string $|G|+|S|+|T|$ used. First bits from binary string $|G|$ used to encode genes, bits of the binary string $|S|$ used to encode observations and bits of the binary string $|T|$ used to encode conditions. When coding, $1$ in the binary string means that the gene, observation or condition is grouped into a bicluster/tricluster while $0$ means that the gene, observation or condition is not grouped in a bicluster / tricluster.

\subsubsection{Local Search}
The local search consists of multiple node deletion, single node deletion and single node addition. Local search is used in the Multi-Objective Evolutionary Algorithm (MOEA) process. The following are the steps of the multiple node deletion algorithm on the matrix $M(I,J)$:

\begin{enumerate}
	\item When $VE^t \geq \delta$ then go to step $2$, otherwise the process is discontinued and gives $M(I,J)$ as the final result of this algorithm.
	\item Delete all the probe ID gene $i\in I$ if it satisfies : $\frac{1}{|J|}\sum_{i\in I}|\hat{b_{ij}}-\hat{\rho_i}| > \alpha \times VE^t $
	\item Recalculate the value of $VE^t$ after deletion.
	\item Delete all of the observation $j \in J$ if it satisfies : $\frac{1}{|I|}\sum_{j\in J}|\hat{b_{ij}}-\hat{\rho_i}| > \alpha \times VE^t$
	\item Recalculate the value of $VE^t$ after deletion.
	\item These results then serve as input to the single node deletion algorithm.
\end{enumerate}

The following are the steps of the single node deletion algorithm:
\begin{enumerate}
	\item Detection of the probe ID gene and observation that has the highest $VE^t$ score in the following ways:
	\item Row score for ID gene to-$i$, $\forall i \in I$, $\mu_{i}=\frac{1}{|J|}\sum_{i \in I}|\hat{b_{ij}}-\hat{\rho_i}|$
	\item Column score for observation to-$j$, $\forall j \in J$,
$\mu_{j}=\frac{1}{|I|}\sum_{j \in J}|\hat{b_{ij}}-\hat{\rho_i}|$
	\item Delete probe ID gene or observation who has the highest score.
	\item Recalculate the value of $VE^t$
	\item Repeat steps $a$ to $c$. If the value of $VE^t \leq \delta $ then stop the iteration
\end{enumerate}

The final result from the node addition algorithm is data subspace $M(I',J')$, where $I' \subseteq I$ and $J'\subseteq J$. Subspace data that was produced from this algorithm is a bicluster with $VE^t\leq\delta$ and  has the maximum size.

\subsubsection{Fitness Function}
In each generation, the chromosomes will go through an evaluation process using a measuring instrument called fitness. The fitness value of a chromosome describes the quality of the chromosomes in the population. This process will evaluate each population by calculating the fitness value of each chromosome and evaluating it until the stop criteria are met. Some of the criteria for stopping are often used, including: stopping at a certain generation, stopping after for several successive generations the highest fitness value has not changed, stopping in $n$ generations does not get a higher fitness value.

In this study, for the first method, namely the search for bicluster with MOEA, the researcher wanted the size of the bicluster that met the large homogeneity criteria and the error value of the minimum bicluster results. From the two criteria above, a formula is formed for the value of the fitness function $f_1$ and $f_2$.

where $|I|$ number of probe ID gene and $|J|$ number of observation in the submatrix or bicluster, $|G|$ and $|S|$ are the number of probe ID genes and number of observations in the preliminary data. $VE^t$ is the Transpose Virtual Error from bicluster while $\delta$ is a given error tolerance limit.

While for the Trigen algorithm, MSL evaluates added to the genetic algorithm fitness function $FF(TRI)$ along with individu size and overlap control, MSL combined with six other factors to be a weighted average. The first three factors are $1-\frac{|TRI_G|}{|D_G|}$, $1-\frac{|TRI_C|}{|D_C|}$, $1-\frac{|TRI_T|}{|D_T|}$ measure the number of gene, condition, and time of $TRI(TRI_{G,C,T})$ compared to the size of the dataset $(|D_{G,C,T}|)$.  Henceforth because MSL minimizes the fitness function, therefore on these three factors are made 1- each proportion to produce a TRI with a larger size when the parameter $w_g$, $w_c$ or $w_t$ is increased. The next three factor $\frac{R_{G}(TRI,SOL)}{|TRI_G|\times|SOL_L|}$,$\frac{R_{C}(TRI,SOL)}{|TRI_C|\times|SOL_L|}$ and $\frac{R_{T}(TRI,SOL)}{|TRI_T|\times|SOL_L|}$ measure the number of gene, condition, and time or depth element $TRI$ on the set of solutions that have been found previously $SOL (|TRI_{G,C,T}|\times|SOL|)$ to produce $TRI$ with a small overlap as the $wo_g$, $wo_c$, or $wo_t$ value increases.
Finally, the main function $\frac{MSL(TRI)}{2\pi}$ measure the $MSL(TRI)$ proportional value close to its maximum value of $2\pi$ to produce $TRI$ with a small MSL value  when the $w_f$ value is increased. The default configuration value for $w_f$, $w_g$, $w_c$, $w_t$, $wo_g$, $w_oc$, and $wo_t$ divided over $w_f$ is $0,8$ and divided by $2$ between $w_g, w_c, w_t, wo_g,wo_c$ and $wo_t$.


\subsubsection{Non Dominant}
An individual can be said to dominate other individuals if it meets the following rules:
\begin{enumerate}
	\item Individu $A$ is no worse off than individu $B$ on all objectives ($A \geq B$).
	\item Individu A has at least one better objective than individu $B$.
\end{enumerate}
Based on these rules, each individu is compared with other individu in a population. The optimization problem in this case is to find the minimum value of 2 objective functions so that the domination condition is if an individu is not worse than another individu and has at least one objective whose value is greater than the other individu. This individu can be in the first front. Then to fill in the next front based on individu who were dominated by the previous front.

\subsubsection{Crowding Distance}
Crowding distance is calculating each individual to measure how close the individual is to his neighbor. The average crowding distance will result in better diversity in the population. The parents are selected from the population using a binary selection tournament based on the crowding distance. Individuals selected in this rank are lower than others or if the crowding distance is greater than others. Selected is the population that produces the offspring of the crossover and mutation operators, which will be discussed in detail in the next section. Populations with current population and current descendants are sorted again by non-dominated sort and only the best $N$ individuals are selected, where $N$ is the population size.

Crowding distance gives the highest value for the solution limit and the average distance of two solutions, that is the solution to-$(i+1)$ and solution to-$(i-1)$ to solution $i$ to each purpose. The crowding distance calculation step for the $i-th$ solution is:

\subsubsection{Crowding Selection Operator}
There are several methods for selecting individuals that are often used, including roulette wheel selection, rank selection, and crowded tournament selection. And in this study, the selection used was the crowded tournament selection. The crowded tournament selection operator is defined as follows: Solution $i$ wins the tournament with another solution that is solution $j$ if one of the following is fulfilled
\begin{enumerate}
	\item Solution $i$ has a better ranking, $r_i < r_j$
	\item if both solutions are on the same front, namely $r_i = r_j$, then the crowding distance of the $i$ solution is greater the crowding distance of the $j$ solution ($CD_i > CD_j$).
\end{enumerate}

\subsubsection{Genetic Operator}
The genetic operators including selection, crossover and mutase are described below:
\begin{enumerate}[label=\alph*)]
	\item Selection: Three groups of individuals are randomly selected in order from lowest to highest according to the fitness function, and then a random selection of the three groups is made. The Cell parameter shows how many of these individuals will pass to the next generation.
	\item Crossover: To complete the next generation, a new individual was created with the following operators: two individuals (parent, $A$ and $B$) are combined to create two new individuals (offspring, child1 and child2). Parents are randomly selected. The $n$-bit inputs in each individual are combined by a random cross point in the gene $TRI_G$, condition $TRI_C$, and time $TRI_T$ and are then swapped to produce two new offspring.
\end{enumerate}
		$mr(t)$ is the row average condition of each tricluster, $mmr$ is row average of all conditions from each tricluster and $|T'|$ is the number of conditions or the depth of the tricluster.

Correlation is the relationship between two vectors. A widely used correlation measure is Pearson's correlation. Pearson's correlation formula based on is for vector mr(t) and   mmr is Mutation: an individual can mutate according to the possibility of mutation. The mutation probability is verified for each individual, and first it is necessary to generate a random number from 0-1 as many as the total gene, i.e. the individual multiplied by the number of $n$-bits and if the random number generated is less than the specified probability of mutation then a mutation will be carried out in the gene, and vice versa if The value generated is more than the mutation probability, so the mutation process is not carried out in the gene. This action is to change the value of the gene if 0 becomes 1, and if the initial value 1 is mutated to 0.


\subsubsection{Multi-Objective Evolutionary Algorithm (MOEA)}
The main steps in the MOEA algorithm are:
\begin{enumerate}
	\item Generates a random size population matrix $P$.
	\item Delete or add rows or columns using local search.
	\item Calculate the fitness function.
	\item Rank the population using dominant criteria.
	\item Calculates the crowding distance value.
	\item Shows the selection results using the crowding tournament selection.
	\item Perform crossover and mutase to produce offspring populations.
	\item Combine the parents and offspring population.
\end{enumerate}

\subsection{Inter Temporal Homogeneity}
Inter temporal homogeneity measures gene homogeneity in various fields of gene observation. For tricluster, inter temporal homogeneity is calculated as follows:

Pearson’s correlation has a value of $ -1 \leq corr(mr(t),mmr) \leq 1$. A correlation of $+1$ means that both are positively linear, while the correlation of $-1$ means that both are negatively linear. Perfect correlation is a correlation that has a value of $0$.

\subsection{$\delta$-Trimax}
The $\delta$-Trimax method is the development of the CC algorithm. This method was developmend by Anirban Bhar by generating triclusters in the form of sub-spaces from 3-dimensional data.This method aims to find triclusters that have a small mean square residual from $\delta$ , where delta is the threshold determined by the researcher.

Suppose that 3D data $Z(A,B,C)$, $M\subseteq Z$, so that $M(I,J,K)$ is sub-space with $I\subseteq A$,$J\subseteq B$ and $K\subseteq C$. Mean square residual(MSR) can obtained using equation \ref{eq:17}


The $\delta$-Trimax methode consists of several algorithms, namely single node deletion, multiple node deletion, node addition and masking. In the single node deletion and multiple node deletion stages, the nodes will be deleted until the MSR is smaller than the specified delta. After deletion is done, it is continued with node addition, to maximize the obtained tricluster volume but still maintaining a small MSR of the delta. Then masking is done to find other triclusters. Flowchart Trimax delta algorithm can be seen in the .

%%%%%%%%%%%%%%%%%%%%%%%%%%%%%%%%%%%%%%%%%%
\section{Experimental Results}
\subsection{THD-Tricluster Implementation}
\subsubsection{THD-Tricluster with New Rresidue Score}
In the generate biclusters stage, for the new residue score, a bicluster search is carried out using a lift algorithm that has two stages, namely single node deletion, and single-node addition, by setting a threshold value of $\delta = 0.08$. Bicluster obtained in different amounts from each condition. The normal condition gets three biclusters, acute gets 100 biclusters, chronic gets 100 biclusters, and non-progressor gets 13 biclusters. In the generate triclusters stage, The tricluster search was carried out in stages, namely by looking for the probe id slices and observation from the bicluster. In the tricluster search process with the results of the bicluster with a new residue score, the minimum probe id $min_p=5$ is determined and the minimum observation $min_o=2$ for the tricluster to be obtained, from this determination 32 tricluster is obtained. The results of all tricluster after being validated through the GPL96 platform downloaded on NCBI showed that 3 genes were known to be associated with HIV-1, namely JUN, ELF-1, and HLA-C.

\subsubsection{THD-Tricluster Transpose Virtual Error}
Biclustering with a modified CC algorithm using transposed virtual error size and parameter $\delta = 0.4$ and $\lambda = 2.5$ has bicluster results that are less than 50. Bicluster for normal conditions is obtained as many as 38 bicluster, bicluster results for acute conditions obtained as many as 31 bicluster, bicluster results for chronic conditions, 49 biclusters were obtained, and the bicluster result for non-progressor conditions was 37 bicluster. The results of this bicluster are then sliced under each condition. In the tricluster search process with the results of the bicluster with transposed virtual error, the minimum probe id $min_p=15$ is determined and the minimum observation $min_o=3$ for the tricluster to be obtained, from this determination, four tricluster is obtained.

The use of transposed virtual errors in the THD Triclustering method succeeded in solving the triclustering problem in 3-dimensional gene expression data, namely by producing four tricluster at a depth of four (normal, acute, chronic, and non-progressor conditions) with an inter-temporal value of 0.9 for each tricluster. This algorithm requires several parameters to work, namely; parameter alpha and delta symbolized by $\alpha$ and $\delta$ in the biclustering process; the Minimum parameter of observation and minimum probe gene symbolized by $m_o$ and $m_p$  in the Triclustering process. The parameters selected in this study use the same value for each condition. Selection of parameters $\alpha = 2.5$ and $\delta = 0.4$ for biclustering search and the minimum probability of the selected gene is 15 with the minimum observation is 4 in the triclustering search process. From the whole research process, four tricluster were generated at a depth of four (normal, acute, chronic, and non-progressor conditions).

\subsubsection{THD-Tricluster Transpose Virtual Error and MOEA}
The Multi-Objective Evolutionary Algorithm method has a stage like NSGA II (Nondominated Sorting Genetic Algorithm II), but an evaluation step is added on that, they are node deletion and node addition. In this case, the researcher used the Transpose Virtual Error($VE^t$). The initial stage in MOEA is population initialization, followed by evaluation using $VE^t$ for the node deletion and node addition stages, the ranking execution process using Non-Dominated Sort, and the selection operation stage using Crowding Distance and Crowding Tournament Selection, and proceed to the operator genetics such as crossover and mutation, continued with recombinant between parents and offspring (children) and re-ranked using Crowding Distance and Crowding Tournament Selection. The final step is to change the parents with the best solution of recombinants. Multi-Objective Evolutionary Algorithm is used to search for bicluster so that the best bicluster is obtained.

In this study, the number of individuals used per iteration was 100, the number of Multi-Objective functions was two and the threshold ($\delta$) selection based on preliminary data Transpose Virtual Error ($VE^t$ ) value. The number of generations that the researcher used was 10 generations with a cross over probability of 0.8 and a mutation probability of 0.1. In this study, two trials were conducted to select the best bicluster for the threshold of 0.5. The results of the Multi-Objective Evolutionary Algoritm cluster are shown.

is an example of a bicluster gene expression graph for nonprogressor conditions. In the  below, it can be seen that the slope of the line between probe id gene shows almost the same slope so that it can be concluded that the shift pattern and scaling pattern of the graph are fulfilled by transpose virtual error detection.

From the results of the bicluster, each condition was looked for a slice between the bicluster so that a tricluster with a depth of two would be obtained. In this study, the results of the tricluster with a depth of two from trials one and two are shown .

The three-state tricluster is the result of a two-state tricluster slice. In this study, the first step was to extract the tricluster with a depth of two, that was extracting the results from the tricluster Uninfected with Acute, Uninfected with Chronic and Acute with Chronic. The two-depth tricluster was obtained by slicing the probe id gene and the samples in the two-depth tricluster. The results of the three depth tricluster are shown in.

From the results of experiment 1, it was obtained tricluster with three conditions, namely uninfected, acute and chronic conditions with 325 probe id genes in observation one and observation two. Whereas in experiment 2, the tricluster obtained in the Uninfected, Acute and Chronic conditions with the probe id genes was 326 in conditions two, three and four. In experiment two with a depth of three, a slice between Acute, Chronic and Nonprogressor was obtained, but only one observation. In this study, there was no tricluster with a depth of 4.

Inter Temporal Homogeneity is to measure the tricluster results obtained. This was an evaluation for each tricluster. From the results of Inter Temporal Homogeneity, it was found that the tricluster results for the two-condition tricluster and the three-condition tricluster were very good because the correlation values for all triclusers were close to one. The following shows the results of the tricluster correlation for experiment one and experiment two.


From experiments 1 and 2 above, it was found that the tricluster obtained was only at depths two and three, while for depth four was not found. Experiment 1 showed that there was no relationship between nonprogressors and the other three conditions, namely uninfected, acute and chronic from the probe ID genes selected through the selection of the Relative Deviation and Absolute Deviation probes ID gene. Meanwhile, for the tricluster with a depth of three, it was found that the probe ID gene 208812\_x\_at contained HLA-C. and probe ID gene 209602\_s\_at containing GATA-3. Based on information obtained from Kakati, et al. HLA-C and GATA-3 are genes related to HIV-1. In the second experiment, the tricluster with a depth of three obtained the same results, namely the probe ID gene 208812\_x\_at contained HLA-C, while another probe ID gene, 201465\_s\_at, contained a gene with the name JUN. JUN also in the THD-tricluster paper shows a relation with HIV-1.

Thus, in this study, a probe ID gene with ID 208812\_x\_at was obtained with the gene name HLA-C, the probe ID gene 209602\_s\_at with the gene name GATA-3 and the probe ID gene 201465\_s\_at with the name JUN. All three of these probe ID gene were associated with HIV-1. In addition to the analysis of gene ID probes obtained , the researchers also took gene ID probe data from the site \url{ https://www.ncbi.nlm.nih.gov/geo/query/acc.cgi?acc=GPL96}.

\subsection{TRIGEN}
Whereas for the Trigen algorithm with a multi slope measure size, the 10 tricluster results obtained are of the size shown
First, it could be seen based  that the resulting tricluster weight size is maximum. It could be seen that the number of probe-id genes, which number has exceeded 50\% of the total probe-id genes in the initial dataset, was 2577. Second, it could also be seen in  that the tricluster solutions did not have a large overlap. It could be seen that only 1 or 2 samples in this study have the same observations, and there were also among all the tricluster results that did not have the same observation coordinates. This showed that the overlap parameters entered are working properly. Then the following is a graphical example of the representation of the results of tricluster 1 in , it could be seen that the variations of the probe-id gene expressions under two samples or observations have almost the same shape for each condition. This showed that the results of the grouping have the same pattern in all conditions.

So, based on the aim of the Trigen algorithm which maximizes the fitness function in which the Trigen fitness algorithm function in this study has been added, the Multi Slope Measure measure has worked very well in grouping data that have the same pattern. And from the analysis of the 10 tricluster results obtained, 6 genes related to HIV were obtained based on the gene bank , namely HLA-C, JUN, CCR5, ELF1, CX3CR1, and GATA-3.


\subsection{$\delta$-Trimax}
A tricluster search of HIV data with $\delta$-Trimax was carried out. By using $\delta =0.0046$ and $\lambda =1.25$, we get 202 triclusters. Next, the tricluster with the smallest TQI is selected. From this tricluster, genes related to HIV-1 were obtained, namely AGFG1, EGR1, and HLA-C.

A comparison of the results of each method can be seen in the table


%%%%%%%%%%%%%%%%%%%%%%%%%%%%%%%%%%%%%%%%%%
\section{Conclusions}

In the THD-Tricluster method. The tricluster results obtained by using a new residue score obtained 32 triclusters which shows genes associated with HIV, namely ELF-1, HLA-C, and JUN. Tricluster results obtained by using the transposed virtual error size obtained 4 triclusers which indicate genes associated with HIV, namely ELF-1 and HLA-C. And when using $\delta$-trimax, obtained 202 tricluster which indicate genes asscociated with AGFG1, EGR1, and HLA-C.

The results of the bicluster are used to build a tricluster by generating tricluster. Tricluster results were obtained for depth two and depth three. The results of the tricluster on the HIV-1 gene expression data showed genes associated with HIV-1, namely HLA-C, GATA-3, and JUN. From the simulation results of the Trigen Algorithm program with multi-slope measure evaluation, the target of 10 triclusers was successfully obtained 10 best triclusters containing HIV-1 biomarkers (HLA-C, JUN, CCR5, ELF1, CX3CR1, and GATA-3) in all conditions such as uninfected, acute, chronic, and nonprogressors. Therefore, from the five methods that have been carried out in this study an HIV biomarker is obtained, namely HLA-C.

\section*{Acknowledgement}
\label{}
NKB-035/UN2.F3/HKP.05.00/2021 is the number of research grant funding by FMIPA Universitas Indonesia.

%% The Appendices part is started with the command \appendix;
%% appendix sections are then done as normal sections
%% \appendix

%% \section{}
%% \label{}

%% References
%%
%% Following citation commands can be used in the body text:
%% Usage of \cite is as follows:
%%   \cite{key}         ==>>  [#]
%%   \cite[chap. 2]{key} ==>> [#, chap. 2]
%%

%% References with BibTeX database:


%% Authors are advised to use a BibTeX database file for their reference list.
%% The provided style IEEEtran.bst formats references is generally used.

%% For references without a BibTeX database:




\end{document}

%%
%% End of file `iaesarticle.tex'. 